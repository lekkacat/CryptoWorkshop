\documentclass[utf8x]{beamer}
\usetheme{Berlin}
\usepackage{movie15}
\begin{document}
\title{Crypto Workshop 0.1}   
\author{Anonym} 
\date{\today} 

\frame{\titlepage} 

\frame{\frametitle{Inhaltsverzeichnis}
\tableofcontents
} 
\section{Freiheit vs. Sicherheit} 
	\frame{
			\frametitle{Freiheit vs. Sicherheit?}
			\tableofcontents[currentsection]
	}
 
	\subsection{Digitale Selbstbestimmung} 
	\frame{\frametitle{Digitale Selbstbestimmung}}

	\subsection{Der Feind} 
		\frame{\frametitle{Der Feind}}

	\subsection{Angriffe auf unsere Selbstbestimmung} 
		\frame{\frametitle{Fährten im Cyperspace}}
		\frame{\frametitle{Services im Netz} }
		\frame{\frametitle{Infrastruktur}
	}

\section{Digitale Selbstverteidigung} 
	\frame{\frametitle{Digitale Selbstverteidigung}
		\begin{itemize}
			\item Kontrolle über Hardware
			\item Kontrolle über Software
			\item Sichere Kommunikation
			\item Sichere Datenspeicherung
			\item Anonymität
			\item Sichere Services
		\end{itemize}
	}
		
\section{Realitätsabgleich}

\subsection{Freie Software - Free Software} 
		\frame{\frametitle{Die 4 Freiheiten}
			"Frei" in "Freie Software" bezieht sich auf Freiheit, nicht auf den Preis. Insbesondere definieren wir vier Freiheiten Freie Software: 
 
			\begin{itemize}
				\item Freiheit 0: Die Freiheit, das Programm für jeden Zweck auszuführen.
				\item Freiheit 1: Die Freiheit, die Funktionsweise eines Programms zu untersuchen, und es an seine Bedürfnisse anzupassen.
				\item Freiheit 2: Die Freiheit, Kopien weiterzugeben und damit seinen Mitmenschen zu helfen.
				\item Freiheit 3: Die Freiheit, ein Programm zu verbessern, und die Verbesserungen an die Öffentlichkeit weiterzugeben, sodass die gesamte Gesellschaft profitiert.
			\end{itemize}
		}

\section{Cryptowars} 
\section{NSA und Cryptocalypse} 
\section{Was ist Verschlüsselung?} 
\section{Geschichte der Verschlüsselung} 
\section{Anonymisierung} 
\section{TOR} 
\section{VPNs} 

\end{document}
